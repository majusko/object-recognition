
%%This is a very basic article template.

%%There is just one section and two subsections.

\documentclass{article}

\usepackage[slovak]{babel}
\usepackage[utf8]{inputenc}

\begin{document}

\title{Rozpoznávanie dobravných značiek}
\author{Mário Kapusta}

\maketitle

\begin{abstract}
V praci sme sa zaoberali
\end{abstract}

\section{Počítačové videnie}
Nejaký obkec o počítačovom videní
\subsection{História počítačového videnia}
Niečo krátke o histórii počítačového videnia
\subsection{Hlavné témy počítačového videnia}
Obkec o rozdelení počítačového videnia a rôznych odvetviach venovania
\subsubsection{Transformácia}
Niečo o trnaformácii.
\subsubsection{Filtrovanie a kompresia}
Niečo o kompresii.
\subsubsection{Vylepšovanie obrazu}
Niečo o vylepšovaní obrazu.
\subsubsection{Rozpoznávanie objektov}
Niečo o rozpoznávaní objektov
\subsubsection{Pozíciovanie}
Niečo o rozpoznávaní poziciovani
\subsection{Technológie}
Niečo o o technológiách rozpoznávania vo všeobecnosti
\subsubsection{OpenCV}
Niečo o opencv - textik k tomu:
http://simplecv.tumblr.com/post/19307835766/opencv-vs-matlab-vs-simplecv
\subsubsection{Matlab}
Niečo o matlabe
\subsubsection{SimpleCV}
Niečo o simplecv

\section{OpenCV, Android a Java}
Obkec o použití knižnice OpenCV a OS Android resp. jazyku Java. 
\subsection{Funkcionalita OpenCV}
budem opisovat najpouzivanejsie funkcie, co robia a ake algoritmy sa v nich pouzivaju
\subsubsection{cvtColor}
\subsubsection{Canny}
\subsubsection{HoughLinesP}
opisat nieco o houghovej transformacii
\subsubsection{GaussianBlur}
\subsubsection{inRange}
\subsubsection{threshold}
\subsubsection{bitmapToMat}
\subsubsection{findContours}
\subsubsection{boundingRect}
\subsubsection{drawContours}
\subsubsection{contourArea}
\subsubsection{fitEllipse}

\section{Návrh riešenia}
\subsection{Návrh algoritmov}
\subsubsection{Návrh algoritmu pre detekciu červenej farby}
\subsubsection{Návrh algoritmu pre detekciu modrej farby}
\subsubsection{Návrh algoritmu pre detekciu kruhov}
\subsubsection{Návrh algoritmu pre detekciu trojuholnikov}
\subsubsection{Návrh algoritmu pre detekciu štvorcov}
\subsection{Návrh objektov - UML}
\subsection{Návrh užívteľského prostredia}

\section{Implementácia}
\subsection{Inštalácia Opencv pre Android}
\subsection{Android aplikácia a GUI}
\subsection{Objekty}
\subsubsection{Trieda 1}
\subsubsection{Trieda 2}
\subsubsection{Trieda 3}

\section{Výsledky aplikácie}
\subsection{Detekcia kruhových značiek}
\subsubsection{Značky modrej farby}
\subsubsection{Značky červenej farby}

\section{Záver}


\end{document}
